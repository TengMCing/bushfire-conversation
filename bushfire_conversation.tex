% Options for packages loaded elsewhere
\PassOptionsToPackage{unicode}{hyperref}
\PassOptionsToPackage{hyphens}{url}
%
\documentclass[
  11pt,
  a4paper,
]{article}
\usepackage{lmodern}
\usepackage{amssymb,amsmath}
\usepackage{ifxetex,ifluatex}
\ifnum 0\ifxetex 1\fi\ifluatex 1\fi=0 % if pdftex
  \usepackage[T1]{fontenc}
  \usepackage[utf8]{inputenc}
  \usepackage{textcomp} % provide euro and other symbols
\else % if luatex or xetex
  \usepackage{unicode-math}
  \defaultfontfeatures{Scale=MatchLowercase}
  \defaultfontfeatures[\rmfamily]{Ligatures=TeX,Scale=1}
\fi
% Use upquote if available, for straight quotes in verbatim environments
\IfFileExists{upquote.sty}{\usepackage{upquote}}{}
\IfFileExists{microtype.sty}{% use microtype if available
  \usepackage[]{microtype}
  \UseMicrotypeSet[protrusion]{basicmath} % disable protrusion for tt fonts
}{}
\makeatletter
\@ifundefined{KOMAClassName}{% if non-KOMA class
  \IfFileExists{parskip.sty}{%
    \usepackage{parskip}
  }{% else
    \setlength{\parindent}{0pt}
    \setlength{\parskip}{6pt plus 2pt minus 1pt}}
}{% if KOMA class
  \KOMAoptions{parskip=half}}
\makeatother
\usepackage{xcolor}
\IfFileExists{xurl.sty}{\usepackage{xurl}}{} % add URL line breaks if available
\IfFileExists{bookmark.sty}{\usepackage{bookmark}}{\usepackage{hyperref}}
\hypersetup{
  pdftitle={Yes, 2019-2020 Victorian bushfires were most likely caused by lightning but there's more to be learned from open data},
  pdfauthor={Di Cook, Weihao Li, Emily Dodwell},
  hidelinks,
  pdfcreator={LaTeX via pandoc}}
\urlstyle{same} % disable monospaced font for URLs
\usepackage[margin=1in]{geometry}
\usepackage{longtable,booktabs}
% Correct order of tables after \paragraph or \subparagraph
\usepackage{etoolbox}
\makeatletter
\patchcmd\longtable{\par}{\if@noskipsec\mbox{}\fi\par}{}{}
\makeatother
% Allow footnotes in longtable head/foot
\IfFileExists{footnotehyper.sty}{\usepackage{footnotehyper}}{\usepackage{footnote}}
\makesavenoteenv{longtable}
\usepackage{graphicx,grffile}
\makeatletter
\def\maxwidth{\ifdim\Gin@nat@width>\linewidth\linewidth\else\Gin@nat@width\fi}
\def\maxheight{\ifdim\Gin@nat@height>\textheight\textheight\else\Gin@nat@height\fi}
\makeatother
% Scale images if necessary, so that they will not overflow the page
% margins by default, and it is still possible to overwrite the defaults
% using explicit options in \includegraphics[width, height, ...]{}
\setkeys{Gin}{width=\maxwidth,height=\maxheight,keepaspectratio}
% Set default figure placement to htbp
\makeatletter
\def\fps@figure{htbp}
\makeatother
\setlength{\emergencystretch}{3em} % prevent overfull lines
\providecommand{\tightlist}{%
  \setlength{\itemsep}{0pt}\setlength{\parskip}{0pt}}
\setcounter{secnumdepth}{5}
%% Any special functions or other packages can be loaded here.

\usepackage[no-weekday]{eukdate}
\usepackage{sourceserifpro}
\usepackage[scaled=0.86]{DejaVuSansMono}
\usepackage{float,bm,setspace}
\setstretch{1.2}

%% CAPTIONS
\RequirePackage{caption}
\DeclareCaptionStyle{italic}[justification=centering]
 {labelfont={bf},textfont={it},labelsep=colon}
\captionsetup[figure]{style=italic,format=hang,singlelinecheck=true}
\captionsetup[table]{style=italic,format=hang,singlelinecheck=true}

%% GRAPHICS
\RequirePackage{graphicx}
\setcounter{topnumber}{2}
\setcounter{bottomnumber}{2}
\setcounter{totalnumber}{4}
\renewcommand{\topfraction}{0.85}
\renewcommand{\bottomfraction}{0.85}
\renewcommand{\textfraction}{0.15}
\renewcommand{\floatpagefraction}{0.8}

%% BIBLIOGRAPHY

\makeatletter
\@ifpackageloaded{biblatex}{}{\usepackage[style=authoryear-comp, backend=biber, natbib=true]{biblatex}}
\makeatother
\ExecuteBibliographyOptions{bibencoding=utf8,minnames=1,maxnames=3, maxbibnames=99,dashed=false,terseinits=true,giveninits=true,uniquename=false,uniquelist=false,doi=false, isbn=false,url=true,sortcites=false}

\DeclareFieldFormat{url}{\texttt{\url{#1}}}
\DeclareFieldFormat[article]{pages}{#1}
\DeclareFieldFormat[inproceedings]{pages}{\lowercase{pp.}#1}
\DeclareFieldFormat[incollection]{pages}{\lowercase{pp.}#1}
\DeclareFieldFormat[article]{volume}{\mkbibbold{#1}}
\DeclareFieldFormat[article]{number}{\mkbibparens{#1}}
\DeclareFieldFormat[article]{title}{\MakeCapital{#1}}
\DeclareFieldFormat[inproceedings]{title}{#1}
\DeclareFieldFormat{shorthandwidth}{#1}
% No dot before number of articles
\usepackage{xpatch}
\xpatchbibmacro{volume+number+eid}{\setunit*{\adddot}}{}{}{}
% Remove In: for an article.
\renewbibmacro{in:}{%
  \ifentrytype{article}{}{%
  \printtext{\bibstring{in}\intitlepunct}}}

\makeatletter
\DeclareDelimFormat[cbx@textcite]{nameyeardelim}{\addspace}
\makeatother
\renewcommand*{\finalnamedelim}{%
  %\ifnumgreater{\value{liststop}}{2}{\finalandcomma}{}% there really should be no funny Oxford comma business here
  \addspace\&\space}
\usepackage[style=authoryear-comp,]{biblatex}
\addbibresource{references.bib}

\title{Yes, 2019-2020 Victorian bushfires were most likely caused by lightning but there's more to be learned from open data}
\author{Di Cook, Weihao Li, Emily Dodwell}
\date{}

\begin{document}
\maketitle

\hypertarget{intro}{%
\section{Introduction}\label{intro}}

In a Jan 11, 2020 ABC article ``The truth about Australia's fires --- arsonists aren't responsible for many this season'' reports about Victoria:

\begin{quote}
``The Country Fire Authority (CFA) said the majority of fires were not arson-related. `Most of the fires have been caused by lightning', said Brett Mitchell, the CFA incident controller in Bairnsdale, in East Gippsland. `Our intelligence suggests there are no deliberate lightings that we are aware of.' Victoria Police had no arson figures available for this bushfire season, but said in the 12 months to September 2019, a dozen people had been arrested for causing bushfires.''
\end{quote}

These words were spoken in response to a fierce war of hashtags on twitter ``\#ClimateEmergency'' vs ``\#ArsonEmergency''. In the ABC article most other states quoted numbers for arson and lightning, but Victoria's response was more vague. This prompted us to dig into open data resources to learn as much as we could about the cause of the bush fires in Victoria.

\hypertarget{data}{%
\section{Data sources and collation}\label{data}}

To track bushfires in Australia remotely with high temporal and spatial resolution, we used hotspot data taken from the Himawari-8 satellite. This data was filtered using the firepower information to keep only hotspots that were most likely fires. We clustered the hotspots in time and geographically, to estimate the ignition point and time of fires, and label fires. This data was supplemented with data from other sources: weather, fuel load distance to camp sites, roads and CFA stations. Weather information included temperature, rainfall, wind, and solar exposure from various sources. Rainfall was aggregated at different time intervals to provide cumulative moisture for locations.

A model was trained to predict one of four causes (lightning, accident, arson, burning off) on the fire origins data from the Victorian Department of Environment, Land, Water and Planning (DELWP), which we supplemented with the same weather, fuel layer, distances data as the hot spot data. The performance of the model was very good: 75\% overall accuracy, 90\% accurate on lightning, 78\% for accidents, 54\% for arson which was mostly confused with accident, as would make sense. The most important contributors to distinguishing between lightning and arson (or accident) ignition were distance to CFA stations, roads and camp sites, and average wind speed over the past 12-24 months. Smaller distances were most likely arson or accident, as might be expected.

\hypertarget{models}{%
\section{Model predictions}\label{models}}

bar chart or table of causes

Maps of predicted causes by month showing arson and accident

\hypertarget{validation-of-predictions}{%
\section*{Validation of predictions}\label{validation-of-predictions}}
\addcontentsline{toc}{section}{Validation of predictions}

Historical data cases histograms

Show somewhere that it looks right?

Our predictions might disagree with this:

French Island likely due to dry lightning
\url{http://www.mpnews.com.au/2020/01/21/all-clear-after-island-fire-fright/}

``Dry lightning is being blamed for the blaze which was thought to have started on the previous Wednesday and smouldered for three days before an east-wind change brought it surging to life.''

zoom into map from app

Link to the app

\hypertarget{acknowledgements}{%
\section{Acknowledgements}\label{acknowledgements}}

This analysis is based on a research paper co-authored by Di Cook and Weihao Li, a Monash University Honours student, from the Department of Econometrics and Business Statistics at the Monash Business School, and Emily Dodwell, previously a member of AT\&T Research in New York City. The full analysis is available at \url{https://github.com/TengMCing/bushfire-conversation}.

The code used for the analysis can be found in the github repository \href{https://github.com/TengMCing/thesis}{github.com/TengMCing/thesis}.

You can explore the historical fire data, predictions for 2019-2020 fires and a fire risk map for Victoria using the shiny app at \href{https://ebsmonash.shinyapps.io/VICfire/}{ebsmonash.shinyapps.io/VICfire/}

\printbibliography

\end{document}
